\documentclass[Main.tex]{subfiles}
\begin{document}
\section{Eerste implementatie}
\subsection{Brute force}
Deze opgestelde boom wordt gebruikt om een vergelijking te vinden passende bij verscheidene gebruikersvoorbeelden. Het invullen van elke mogelijkheid uit de boom met de gegeven variabelen van de gebruiker heeft een complexciteit van $b^{d}*v^{d+1}$. Voor elke E symbool moet elke mogelijkheid ingevuld worden om de volledige term te laten evalueren. Indien deze term evalueert tot de gezochte doelwaarde voor het voorbeeld, dan is dit een mogelijke vergelijking voor het gebruikers probleem. \\
Indien dat deze vergelijking dan voldoet aan de overige voorbeelden is dit een oplossing voor de gebruiker en is het probleem dus opgelost. En is het mogelijk om de overige gebruikersdata te vervolledigen aan de hand van deze gevonden vergelijking.
%TODO Voorbeeldje geven van de werking in stappen? Of in stappen uitleggen in plaats van tekst.

\subsection{Gebruikersdata}
De meeste applicatie worden geschreven om een bepaald gebruikersprobleem op het lossen. Hiervoor bestaat dus al de nodige gebruikersdata waaraan het algoritme zich kan testen. In het geval van dit probleem is bestaat er geen gebruikersdata, omdat er momenteel nog geen belangstelling is voor deze techniek. Alle data waarop experimenten zullen uitgevoerd worden zullen aan de hand van zelfgegenereerde data zijn. De manier waarop deze data genereerd wordt is gestructureerd met een zekere willekeur op het bepalen van kleine variaties.

\end{document}