\documentclass[Main.tex]{subfiles}
\begin{document}
\section{Probleemstelling}
Gebruikers hebben reeds de mogelijkheid om patroonherkenning te gebruiken op hun dataset. Toch kan de gebruiker op zoek zijn naar een wiskundige relatie tussen in zijn dataset in plaats van een patroon.
\subsection{Context-vrije grammatica}
Om deze wiskundige relaties op te bouwen wordt er gebruikt gemaakt van een contextvrije grammatica. Volgens de 'formele taal theorie' is een contextvrije grammatica een formele grammatica waarbij elke productieregel er uit ziet als volgt: $V \rightarrow w$. Hier staat $V$  voor \'e\'en niet-terminaal symbool is en $w$ een tekenreeks is van niet-terminale en terminale symbolen. 'Contextvrij' betekend dat de productieregels kunnen toegepast worden los van de context waarin het niet-terminale symbool zich bevindt. \\

\begin{framed}
\begin{center}
$E \rightarrow E + E$ \\
$E \rightarrow E - E$  \\
$E \rightarrow E \times E$ \\
$E \rightarrow E \div E$ \\
$E \rightarrow 1,2,\dotsc,9,a,b,\dotsc$
\end{center}
\end{framed}

Als een andere wiskundige operatie gewenst is moet deze toegevoegd worden als een productieregel aan bovenstaande grammatica. De gemakkelijke aanpasbaarheid is een voordeel van het werken met een context-vrije grammatica. Omdat de productieregels volledige onafhankelijk zijn van elkaar is het eenvoudig om de gewenste operaties te defini\"eren.

\subsection{Boomstructuur}

Door herhaaldelijke toepassing van de productieregels onstaat er een boomstructuur. In deze boom bevinden zich alle mogelijke samenstellingen van productieregels die door de grammatica bepaald worden. Deze operatie zorgt voor een exponenti\"ele groei $r^{d}$ waarbij $d$ staat voor de diepte van de boom en $r$ voor het aantal mogelijke productieregels van de vorm $E \rightarrow E$  $operand$ $ E$. De groei van deze boom staat beschreven in volgende figuur en tabel.

\includegraphics[width=\columnwidth]{treeSize.png}
%TODO Tabel is overkill?
%\begin{center}
%\begin{tabular}{| l | r |}
%\hline
%	Diepte & Elementen \\ \hline \hline
%	1 &	1 \\ \hline
%	2 &	4 \\ \hline
%	3 &	16 \\ \hline
%	4 &	64 \\ \hline
%	5 &	256 \\ \hline
%	6 &	1024 \\ \hline
%	7 &	4096 \\ \hline 
%	8 &	16384 \\ \hline
%\end{tabular}
%\end{center}

Aangezien de groei van de boom exponentieel is bekomt men veel grotere berekeningstijden volgens toename van de diepte in de boom. De grote overhead die het opstellen van de boom is, heeft tot gevolg dat het ten zeerste aangeraden om deze boom slechts \'e\'enmalig op voorhand uit te werken.

\subsection{Evaluatiecriteria}

'Snelheid, correctheid, gebruikersgemak en uitbreidbaarheid' zijn de vier evaluatie criteria waarop het onderzoek gaat beoordeeld worden. Het doel van het onderzoek is dus een effici\"ent, correct en uitbreidbaarheid algoritme te vinden dat uit een gegeven set van voorbeelden een geschikte vergelijking kan vinden. Ook het gebruikersgemak is van belang bij het het defini\"eren van wat de gebruiker juist van het algoritme verwacht.
\end{document}