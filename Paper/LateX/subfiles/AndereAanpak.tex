\documentclass[Main.tex]{subfiles}
\begin{document}
\section{Limieten van het onderzoek}
Wegens verscheidene redenen zijn een aantal methoden niet onderzocht. Hieronder is een overzicht te vinden van deze methoden en de verklaring waarom deze paden niet bewandeld zijn. 

\subsection{Heuristiek}
Een mogelijk hulpmiddel bij het verbeteren van zoekalgoritmes is het implementeren van een passende heuristiek. In het onderzoek naar een heuristiek voor dit project is gebleken dat de waarde van knopen enorm snel kunnen verschillen, zoals de invloed van $\times$, $\div$ en \^{}. Om de oplossingsgraad die op dit moment bereikt wordt te bewaren is het onmogelijk een goede heuristiek te vinden. Omdat er in dit onderzoek vanuit gegaan is dat de gebruiker geen extreem lange vergelijkingen zoekt, zal de heuristiek bijgevolg ook geen meerwaarde bieden. Mocht later blijken dat er vraag is naar het vinden van grotere vergelijkingen zal deze denkpiste opnieuw geopend worden.

\subsection{Andere talen}
Er is een poging gewaagd het project te implementeren met behulp van een programmeertaal dan Java. Door onze beperkte kennis van Prolog en Haskell zijn we hier echter niet in geslaagd. De problemen bevonden zich vooral in het prunen van de boom. We zijn geslaagd in het implementeren van het brute force algoritme. De snelheid van het huidige algoritme was echter hoger. Er is dan ook voor gekozen dit pad niet verder te bewandelen.

\subsection{Op basis van onbekende}
Elke vergelijking met \'e\'en onbekende variabele is oplosbaar. Indien er meerdere vergelijkingen zijn dan mogen er ook meerdere variabelen onbekend zijn en blijft het stelsel van deze vergelijkingen oplosbaar. Problemen hierdoor zijn dat bij het toepassen van dit principe op een kleine set van vergelijkingen er een enorme hoeveelheid van mogelijke oplossingen gevonden worden. In deze set van oplossingen bevinden zich een groot deel van vergelijkingen die gebruik maken van constanten en dus naar grote waarschijnlijkheid niet zullen geschikt zullen zijn voor verder voorbeelden. De extra controle die nodig is om dit soort van zaken na te gaan is een te grote overhead ten opzichte van de rest van het programma. Dit leek geen goede piste voor het huidige onderzoek.

\end{document}