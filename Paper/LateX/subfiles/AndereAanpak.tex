\documentclass[Main.tex]{subfiles}
\begin{document}
\section{Limieten van het onderzoek}
Een aantal onderzoeksmethoden zijn niet gevolgd om uiteenlopende redenen. Hieronder is een overzicht te vinden van deze methoden en een verklaring waarom deze paden niet bewandeld zijn geweest. 

\subsection{Heuristiek}
Een mogelijk hulpmiddel bij het verbeteren van zoekalgoritmes is het implementeren van een passende heuristiek. In dit onderzoek is gebleken dat de waarde van de knopen enorm snel kan verschillen, zoals door invloed van $\times$, $\div$ en \^{}. Om de oplossingsgraad die bereikt wordt te bewaren, is het onmogelijk een goede heuristiek te vinden. In dit onderzoek werd verwacht dat de gebruiker geen extreem lange vergelijkingen zoekt. Hierdoor zal de heuristiek bijgevolg ook geen meerwaarde bieden. Mocht later blijken dat er vraag is naar het vinden van grotere vergelijkingen zal deze denkpiste opnieuw geopend moeten worden.

\subsection{Andere talen}
Er werd gepoogd het project te implementeren met behulp van een andere programmeertaal dan Java. Door onze beperkte kennis van andere programmeertalen zijn we hier echter niet in geslaagd. De problemen bevonden zich vooral in het prunen van de boom, het implementeren van een 'brute force algorithm' is wel gelukt. De snelheid van dit algoritme was hoger. Dit pad werd daarom niet verder bewandeld.
%TODO Snelheid???

\subsection{Op basis van onbekende}
Elke vergelijking met \'e\'en onbekende variabele is oplosbaar. Indien er meerdere vergelijkingen zijn, dan mogen er ook meerdere variabelen onbekend zijn en blijft het stelsel van deze vergelijkingen oplosbaar. Het toepassen van dit principe op een kleine set van vergelijkingen creert een probleem. Er wordt een enorme hoeveelheid van mogelijke oplossingen gevonden. In deze set van oplossingen bevinden zich een groot deel van vergelijkingen die gebruik maken van te veel constanten en zullen naar alle waarschijnlijkheid niet geschikt zijn voor verder gebruik. De extra controle, die nodig is om dit soort van zaken na te gaan, is een te grote overhead ten opzichte van de rest van het programma. Dit is geen goede piste voor het huidige onderzoek.

\end{document}