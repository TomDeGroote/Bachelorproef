\documentclass[Main.tex]{subfiles}
\begin{document}
\section{Experimenten}
Voor het verder verklaren van de experimenten zijn er een aantal zaken die vermeld moeten worden. De volgorde waarin de experimenen zijn uitgevoerd is van belang. Bij de meeste experimenten wordt er gebaseerd op de resultaten van de voorgaande experimenten. Er zijn verschillende beoordelingswijzen die gebruikt worden bij het onderzoeken van de experimenten. Enerzijds testen de experimenten op oplossingsgraad\footnotemark[\ref{note:oplossingsgraad}] en anderzijds op de tijdsmarge die het algoritme nodig heeft. 
\par
Indien er niet gebaseerd wordt op de resultaten van vorige experimenten, dan zijn sommige van de latere experimenten enorm tijdrovend. Deze tijdrovende experimenten tonen ook niets aan wat niet aangetoond kan worden met de huidige manier van experimenteren.
\par
Alle experimenten zijn uitgevoerd op een '2.26 GHz Intel Core 2 Duo' processor. Het aantal iteraties van elk experiment is 500 en de gebruikte diepte van de boom is tot en met 5 termen breed, een voorbeeld van een vergelijking op die diepte is $K1^{K2}+K3 \div K4*K5$.

\subsection{Boom op voorhand opstellen}

Enkel tijdswinst aanhalen hier %TODO

\subsection{Prunen op vergelijkingen}


Boom wordt niet op voorhand opgesteld aangezien we dan meer kunnen prunen, met name K1-K1 enzo
\subsection{Gewichten}
Voor het bepalen van een optimale set van constante waarden, die gebruikt kunnen worden door het algoritme, wordt er op dezelfde input verscheidene sets van constanten getest. 


Uitgevoerd op basis van een geprunde boom (experiment hierboven)
\subsection{Prunen op vergelijkingen en gewichten}
Gebasseerd op prime gewichten, boom wordt niet meer op voorhand berekend
\end{document}