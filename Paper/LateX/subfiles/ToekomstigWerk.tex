\documentclass[Main.tex]{subfiles}
\begin{document}
\section{Toekomstig werk}
Door een beperkte tijd is het onmogelijk om al het onderzoek/werk te doen dat kan gedaan worden in verband met dit onderwerp. Hieronder worden een mogelijkheid voor de toekomst overlopen.

\subsection*{Toevoegen haakjes}
Door het toevoegen van haakjes kunnen bepaalde vergelijkingen op een hoger niveau gevonden worden, zoals bv. $(A+B)^2$ wordt voorlopig pas gevonden als $A^{2}+2 \ast A \ast B+B^{2}$. Het verschil in diepte tussen deze twee termen is 3, want $(A+B)^2$ wordt gevonden op niveau 4 terwijl $A^{2}+2AB+B^{2}$ pas op diepte 7 gevonden wordt. Nadelig aan het toevoegen van haakjes is dat er een extra productieregel aan de contextvrije grammatica moet worden toegevoegd (nl. $E \rightarrow (E)$). Dit zorgt voor een stijging van het aantal knopen en dus een bredere zoekruimte. 

\par De huidige manier van prunen maakt de implementatie van deze productieregel moeilijk, dit komt doordat context vrij programmeren in Java niet volledig haalbaar is. Een mogelijkheid is om de regel $E \rightarrow (E operand E)$ te implementeren, maar door de meeste rechtse invulling die gebruikt word in dit programma zijn er dan bepaalde zaken die niet gevonden kunnen worden. Tenslotte moet de uitkomst van de vergelijkingen kunnen worden berekend. Het toevoegen van extra bewerkingen is echter wel triviaal. Er moet slechts \'e\'en klasse worden toegevoegd die de abstracte klasse symbol implementeerd.

\end{document}