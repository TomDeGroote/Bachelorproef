\documentclass[Main.tex]{subfiles}
\begin{document}
\section{Toekomstig werk}
Door bepaalde omstandigheden was het onmogelijk om al het onderzoek/werk te doen dat er gedaan kan worden in verband met dit onderwerp. Een aantal zaken zaken waarover reeds nagedacht was, maar wegens tijdgebrek uiteindelijk niet verder onderzocht zijn staan hieronder vermeld.

\subsection{Toevoegen haakjes}
Door het toevoegen van haakjes kunnen bepaalde vergelijkingen op een hoger niveau gevonden worden, zoals bv. $(A+B)^2$ word voorlopig pas gevonden als $A^{2}+2AB+B^{2}$. Het verschil van diepte in de boom tussen deze twee termen is 3, want $(A+B)^2$ wordt al reeds na 4 stappen gevonden terwijl $A^{2}+2AB+B^{2}$ pas op diepte 7 van de boom gevonden wordt. Nadelig aan het toevoegen van haakjes is wel dat er een extra productieregel aan de contextvrije grammatica zou toegevoegd worden (nl. $E \rightarrow (E)$). Een stijging van het aantal knopen op elk niveau is wel en gevolg van deze toevoeging.

Er zal dus een afweging gemaakt moeten worden tussen de kleinere diepte waarop er een geschikte vergelijking gevonden wordt en de breedte van elke diepte die toeneemt afhankelijk van het aantal productieregels die in de contextvrije grammatica aanwezig zijn.

\subsection{Berekenen van 1 losstaand gewicht}
Stel dat de laatste variabele berekent wordt uit de gegeven voorbeelden, dan kan er ook weer op reeds een veel lager niveau een geschikte vergelijking gevonden worden. Een gevolg hiervan is dan wel dat indien er meerdere voorbeelden zijn waarop het algoritme zich moet baseren er veel passende vergelijkingen voor het eerste voorbeeld bij de overige voorbeelden niet zal kloppen en daar dus een bepaalde overhead zal cre\"eren. Het verwachte resultaat is hier dat deze maatregel voordelig is voor vergelijkingen waar er maar \'e\'en voorbeeld is waarop er gebaseerd moet worden. Indien er meerdere voorbeelden zijn waar er op gebaseerd moet worden zal deze maatregel meer overhead veroorzaken dan effectief een verbetering zijn voor het algoritme. 

%TODO Een voorbeeldje bij geven over wat we hier exact bedoelen? Nogal moeilijk om in tekst goed uit te leggen.

\end{document}