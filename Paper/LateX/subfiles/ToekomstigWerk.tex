\documentclass[Main.tex]{subfiles}
\begin{document}
\section{Toekomstig werk}
Wegens tijdsgebrek is het onmogelijk om al het onderzoek/werk te doen dat uitgevoerd kan worden in verband met dit onderwerp. Hieronder wordt een voorbeeld voor de toekomst aangehaald.

\subsection*{Toevoegen haakjes}
Door het toevoegen van haakjes kunnen bepaalde vergelijkingen op een hoger niveau gevonden worden, zoals bv. $(A+B)^2$ wordt in het huidige algoritme pas gevonden als $A^{2}+2 \ast A \ast B+B^{2}$. Het verschil in diepte tussen deze twee termen uit het voorbeeld is drie, want $(A+B)^2$ bevindt zich op diepte vier ($E \rightarrow E^{2} \rightarrow (E)^{2} \rightarrow (E+E)^{2}$) terwijl $A^{2}+2AB+B^{2}$ pas op diepte zeven staat. Nadelig aan het toevoegen van haakjes is dat er een extra productieregel aan de contextvrije grammatica moet worden toegevoegd (nl. $E \rightarrow (E)$). Dit zorgt voor een stijging van het aantal knopen en dus een bredere bewerkingsboom, wat een bredere zoekruimte tot gevolg heeft. 

\par De implementatie van deze productieregel ligt moeilijk, omdat in onze implementatie van het berekenen van knopen geen rekening gehouden wordt met het bestaan van haakjes. De huidige manier van berekenen volgt enkel de volgorde van bewerkingen. Er bestaat een mogelijkheid om productieregel $E \rightarrow (E$ operand $E)$ te implementeren. Maar door de toevoeging van dit soort van productieregels voor elke operatie, zou de breedte van de boom enorm toenemen. Mits aanpassing van de manier waarop knopen berekend worden zou het perfect mogelijk moeten zijn om haakjes toe te laten in de gebruikte grammatica.

\end{document}