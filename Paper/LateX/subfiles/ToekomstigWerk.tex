\documentclass[Main.tex]{subfiles}
\begin{document}
\section{Toekomstig werk}
Wegens tijdsgebrek is het onmogelijk om al het onderzoek/werk te doen dat uitgevoerd kan worden in verband met dit onderwerp. Hieronder wordt een voorbeeld voor de toekomst aangehaald.

\subsection*{Toevoegen haakjes}
Door het toevoegen van haakjes kunnen bepaalde vergelijkingen op een hoger niveau gevonden worden, zoals bv. $(A+B)^2$ wordt in het huidige algoritme pas gevonden als $A^{2}+2 \ast A \ast B+B^{2}$. Het verschil in diepte tussen deze twee termen uit het voorbeeld is 3, want $(A+B)^2$ bevindt zich op diepte 4 terwijl $A^{2}+2AB+B^{2}$ pas op diepte 7 staat. Nadelig aan het toevoegen van haakjes is dat er een extra productieregel aan de contextvrije grammatica moet worden toegevoegd (nl. $E \rightarrow (E)$). Dit zorgt voor een stijging van het aantal knopen en dus een bredere bewerkingsboom, wat een bredere zoekruimte tot gevolg heeft. 

\par De huidige manier van prunen maakt de implementatie van deze productieregel moeilijk. Dit komt doordat contextvrij programmeren in Java niet volledig haalbaar is. Er bestaat een mogelijkheid om productieregel $E \rightarrow (E operand E)$ te implementeren. Door de meest rechtse invulling van de knopen te gebruiken in dit programma worden echter bepaalde vergelijkingen niet gevonden. Het toevoegen van extra bewerkingen is echter wel triviaal. Er moet slechts \'e\'en klasse worden toegevoegd in de code die de abstracte klasse \textit{Symbool} implementeert.

\end{document}