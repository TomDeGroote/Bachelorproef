\documentclass[Main.tex]{subfiles}
\begin{document}
\section{Introductie}
Steeds vaker wordt er van een computer verwacht dat deze complexere taken kan uitvoeren. Terwijl het voor de gebruikers meestal duidelijk is wat er gewenst wordt, is dit voor de computer niet altijd het geval. Een recent voorbeeld hiervan is de 'FlashFill' functionaliteit die momenteel aanwezig is in Excell 2013. Hierbij is het de bedoeling dat de computer het door de gebruiker gewenste patroon op een set van data herkent en kan toepassen op gelijkaardige sets van data om de gebruiker tijd te besparen. \cite{flashFill}

\begin{figure}[!htb]
\centering
\begin{framed}
\begin{tabular}{| c | c | c |}
\hline
Naam & Voornaam & Initialen \\ \hline
Craps & Jeroen & C.J. \\ \hline
De Groote & Tom &  \\ \hline
\end{tabular} \\
$\downarrow$ \\
\begin{tabular}{| c | c | c |}
\hline
Naam & Voornaam & Initialen \\ \hline
Craps & Jeroen & C.J. \\ \hline
De Groote & Tom &  D.G.T \\ \hline
\end{tabular}
\end{framed}
\caption{Voorbeeld Flash Fill}
\label{fig:flashfill}
\end{figure}

\par Een ontbrekende functionaliteit hierin is het herkennen van wiskundige bewerkingen of functies. Met andere woorden, het 'bepalen' van een passende vergelijking voor de gebruikers data. Het verbeteren van het gebruikersgemak bij het bepalen van geschikte wiskundige vergelijkingen is het doel van dit werk. \par  %TODO Key Results

Gelijkaardige functionaliteit aan die van Flash Fill op het vlak van wiskundige vergelijkingen moet dus onderzocht worden. Het onderzoek heeft zijn plaats in het domein van 'Artifici\"ele Intelligentie', meer bepaald 'Machine Learning By Example'. Slechts een zeer beperkte groep van gebruikers hebben dit probleem. Vandaar dat er nog weinig tot geen onderzoek gedaan is met betrekking tot dit onderwerp. Bijgevolg is er geen dataset die kan gebruikt worden als invoer voor de experimenten van het onderzoek.  \par

Toch is het bepalen van dit soort van vergelijking geen volledig onbekend onderzoeksdomein. In 2002 verscheen er een paper die een oplossing gaf voor het \textit{Countdown Problem} \cite{countdown}. Countdown is een Brits televisie programma waarbij de deelnemers een wiskunde vergelijking moeten vinden met de gegeven set van getallen om zo dicht mogelijk tot bij een bepaald streefdoel te komen. Een belangrijke beperking op dit probleem is wel dat elk gegeven cijfer maximaal \'e\'enmalig mag voorkomen. Indien een gebruiker bepaalde waarde meermaals wilt gebruiken is dit niet meer volgens de regels van Countdown,maar aangezien een mogelijke gebruiker deze wel meermaals wil kunnen gebruiken moet er een alternative manier van aanpak gevolgd worden. In dit onderzoek wordt er rekening gehouden met dit soort van gebruikers.
\end{document}