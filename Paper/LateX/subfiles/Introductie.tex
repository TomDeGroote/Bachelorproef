\documentclass[Main.tex]{subfiles}
\begin{document}
\section{Inleiding}
Steeds vaker wordt er van een computer verwacht dat deze meer en meer complexe taken kan uitvoeren. Terwijl het voor de gebruikers meestal duidelijk is wat er gewenst wordt, is dit voor een computer niet altijd het geval. Een recent voorbeeld hiervan is de \textit{FlashFill} \cite{flashFill} functionaliteit die momenteel beschikbaar is in Excel 2013. Hierbij is het de bedoeling dat een computer het door de gebruiker gewenste patroon op een set van data herkent en kan toepassen op gelijkaardige sets van data om de gebruiker tijd te besparen.

\begin{figure}[!htb]
\centering
\begin{framed}
\begin{tabular}{| c | c | c |}
\hline
Naam & Voornaam & Initialen \\ \hline
Craps & Jeroen & C.J. \\ \hline
De Groote & Tom &  \\ \hline
\end{tabular} \\
$\downarrow$ \\
\begin{tabular}{| c | c | c |}
\hline
Naam & Voornaam & Initialen \\ \hline
Craps & Jeroen & C.J. \\ \hline
De Groote & Tom &  D.G.T \\ \hline
\end{tabular}
\end{framed}
\caption{Voorbeeld Flash Fill}
\label{fig:flashfill}
\end{figure}

\par Een ontbrekende functionaliteit bij \textit{FlashFill} is het herkennen van wiskundige bewerkingen of functies, met andere woorden: het vinden van een passende vergelijking voor de data van de gebruiker. Het doel van het werk is het verbeteren van het gebruikersgemak bij het bepalen van geschikte wiskundige vergelijkingen. Een gelijkaardige functionaliteit, zoals deze in \textit{FlashFill}  beschikbaar, moet dus onderzocht worden op het vlak van wiskundige vergelijkingen. Het vinden van dit soort wetmatigheden in de vorm van wiskundige vergelijkingen is een onderdeel van het onderzoeksdomein van \textit{Equation Discovery} \cite{equationDisc}. Dit onderzoek behoort tot het domein van \textit{Artifici\"ele Intelligentie}, meer bepaald \textit{Machine Learning}. Slechts een zeer beperkte groep van gebruikers ondervinden een gelijkaardig probleem. Vandaar dat hierop zo goed als geen onderzoek gedaan is. Er is dan ook geen dataset beschikbaar die kan gebruikt worden als invoer voor de experimenten van dit onderzoek.  \par

Toch is het bepalen van vergelijkingen geen volledig onbekend onderzoeksdomein. In 2002 verscheen er immers een paper die een oplossing gaf voor het \textit{Countdown Problem} \cite{countdown}. Countdown is een Brits televisie programma waarbij de deelnemers een wiskunde vergelijking moeten vinden met een gegeven set van getallen om zo dicht mogelijk tot een bepaald streefdoel te komen. Een belangrijke beperking is wel dat elk gegeven cijfer maximaal \'e\'enmalig mag voorkomen. Indien een gebruiker een bepaalde waarde meermaals wilt gebruiken, volgt dit niet meer de regels van Countdown. Aangezien een mogelijke gebruiker een waarde wel meermaals wil kunnen gebruiken, moet er een alternatieve manier van aanpak gevolgd worden. In dit onderzoek wordt er rekening gehouden met dit soort van gebruikers.
\end{document}