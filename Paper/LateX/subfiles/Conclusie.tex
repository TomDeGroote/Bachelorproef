\documentclass[Main.tex]{subfiles}
\begin{document}
\section{Conclusie}
In deze conclusie zullen de hypotheses kort overlopen worden. Er wordt gestart bij de subhypotheses om vervolgens te eindigen met de hoofdhypothese.
\subsubsection{Er zijn redundante knopen in de originele boom}
Zoals verwacht zijn er inderdaad redundante knopen. Meer nog het aantal redundante knopen vermeerderd naarmate de bewerkingsboom verder wordt uitgewerkt. Een logisch gevolg van het feit dat absoluut redundante knopen ervoor zorgen dat alle kinderen ook redundant zijn. Als er gekeken wordt naar het procentueel aantal redundante knopen per niveau in de bewerkingboom, stagneerd dit rond 35\% redundantie.
\subsubsection{Toevoeging van constanten heeft een postieve invloed}
Het toevoegen van constanten verhoogt de oplossingsgraad. De oplossingsgraad stijgt namelijk van 52\% naar 97\%. Dit is een gevolg van de gebruikte randomgenerator. Deze voorziet namelijk een random gekozen aantal constanten (aantal = 0, 1 of 2). Er wordt verwacht dat de gebruiker zulke vergelijkingen zoekt. Denk maar bijvoorbeeld aan de vergelijking $D = B^{2} - 4 \ast A \ast C$. Het nadeel aan het toevoegen van constanten is echter dat de zoektijd vergroot en er vaak minder diep in de bewerkingsboom kan gezocht worden. Dit probleem wordt besproken in de volgende subhypothese.
\subsubsection{Er bestaat een goede afweging tussen de oplossingsgraad en de benodigde tijd bij het toevoegen van constanten}
De zoektijd neemt toe naarmate er meer constanten worden toegevoegd. Dit is logisch aangezien de uitwerking van de bewerkingsboom breder wordt. Het opzet is dus enerzijds het aantal constanten te beperken en anderzijds effici\"ent om te gaan met deze constanten. Het eerste wordt bereikt door enkel priemgetallen toe te laten kleiner dan tien. Uit experimenten bleek dat de getallen $1, 2, 3, 5, 7$ de hoogste oplossingsgraad leveren zonder de tijd al te veel te laten toenemen. Een mogelijke verklaring hiervoor is dat met deze constanten ook makkelijk andere constanten gemaakt kunnen worden. Het tweede wordt bereikt door neutrale constanten van bewerkingen niet toe te laten. Zo wordt er vermeden dat er op vergelijkingen zoals $1*a$ wordt verdergebouwd. (zo wordt bv $1*a*b$ nooit berekend). Ook wordt er niet toegelaten dat er met constanten andere bestaande constanten worden gegenereerd. Bijvoorbeeld $1+2$ is niet toegelaten aangzien $3$ reeds bestaat. Deze beperkingen lijken klein maar naarmate de bewerkingsboom verder wordt uitgewerkt vergroot de besparing exponentieel.
\subsubsection{Het vermijden van redundante uitwerkingen heeft een positieve invloed op de tijd}
Ook hier kan de hypothese positief beantwoord worden. Dit komt doordat rendundante knopen op een effici\"ente manier worden verwijderd. Er is gekozen om bepaalde redundanties niet te verwijderen omdat het controleren hierop meer tijd vraagt dan de verdere uitwerking. Anderzijds is er gekozen om veralgemeniseringen van bepaalde redundanties te gebruiken omdat de specifieke gevallen opnieuw te veel tijd vragen om weg te werken. Het gevolg hiervan is dat er de oplossingsgraad een beetje daalt maar de zoektijd enorm verminderd. Doordat de zoektijd verminderd is, kunnen er meer vergelijkingen worden bekeken binnen dezelfde tijd. Wat ervoor zorgt dat de oplossingsgraad uiteindelijk toch lichtjes groter is.
\subsubsection{Er kan een passende vergelijking gevonden worden binnen een beperkte tijdspanne.}
Dit is de hoofdhypothese. Enerzijds moet er gemeld worden dat voor vergelijkingen van lengte vijf of kleiner bv. $B^{2} - 4 \ast A \ast C$ er in $87\%$ van de gevallen een oplossingen gevonden wordt. De vergelijkingen die niet gevonden worden bevatten vaak \'e\'en constante die niet tussen de gekozen constanten zit. Het gevolg hiervan is dat de vegelijking pas een niveau dieper in de boom zal gevonden worden. Dit niveau wordt door de beperkte zoektijd niet bereikt. Anderzijds is het resultaat teleurstellend. De oplossingsgraad van het brute force algoritme ligt namelijk op $83\%$ en verschilt dus niet veel van de effi\"ente manier. Dit valt te verklaren door het feit dat zowel het brute force algoritme en het effici\"entere algorimte binnen de twee seconden ongeveer dezelfde zoekruimte afgaan. Het verschil tussen de twee algoritmes is pas groot op verdere niveaus. Jammer genoeg zal naar de gebruiker toe dit verschil dus amper merkbaar zijn. De eindconclusie is: "Het mogelijk is om vergelijkingen van beperkte lengte te vinden binnen een beperkte tijd van twee seconden. Echter zal de oplossingsgraad van het effici\"ente algoritme niet veel verschillen van brute force algoritme".
\end{document}