\documentclass[Main.tex]{subfiles}
\begin{document}
\section{Aanpak}
Omdat de beperkingen van het 'Countdown' probleem niet van toepassing zijn, kunnen een groot deel van de gekende optimalisaties niet gebruikt worden. Er is met verschillende algoritmes geëxperimenteerd. Elk algoritme zal besproken worden aan de hand van enkele stappen.
\subsection{Brute Force}
\subsubsection*{Het opstellen van de bewerkingsboom}
Dit wordt bereikt door gebruik te maken van de context-vrije grammatica. Er wordt gestart met $E$ vervolgens wordt de boom uitgebreid aan de hand van de regels van de CFG. De regel waarbij $E$ echter waardes krijgt ($E \rightarrow 1,2,\dotsc,9,a,b,\dotsc$) wordt pas in de volgende stap gebruikt. %TODO afbeelding bewerkingsboom% 
\subsubsection*{Het uitwerken van de vergelijkingen}
In deze stap wordt de regel $E \rightarrow a,b,\dotsc$ gebruikt om elke vergelijking in de bewerkingsboom een waarde te geven. Enkel deze regel wordt gebruikt en elke mogelijkheid bij elke vergelijking wordt afgegaan. %TODO afbeelding uitgewerkte bewerkingsboom%
\subsubsection*{Het resultaat}
Zolang er vergelijkingen worden gezocht waarin geen constanten voorkomen vind dit algoritme in theorie altijd een oplossing. Het is echter wel zeer inefficiënt, zeker naarmate het aantal kolomwaarden  %TODO andere naam?%
toeneemt.
\subsection{Toevoegen van gewichten}
Ons doel %TODO geen ons gebruiken?%
is het vinden van een vergelijking die de gebruiker verlangt. Aangezien een vergelijking zoals $2 \times x+1$ vaak voorkomt moeten er ook constanten toegelaten worden. Om dit te bereiken manier moet er slechts één wijziging plaatsvinden in het vorige algoritme. Namelijk bij het uitwerken van de vergelijkingen moeten we de regel $E \rightarrow 1,2,\dotsc,9,a,b,\dotsc$ gebruiken in plaats van $E \rightarrow a,b,\dotsc$.
\subsubsection*{Het resultaat}
Het gevolg is dat de uitwerkingstijd van de bewerkingsboom sterk toeneemt. De oplossingsgraad %TODO ander woord%
neemt echter ook sterk toe. De optimale verhouding tussen deze twee wordt bereikt door de keuze van de juiste constanten. Deze keuze zal uit later uit experimenten blijken. Dit algoritme zal gebruikt worden als benchmark.
\subsection{Prunen}
\subsubsection*{Prunen in de bewerkingsboom}
In de bewerkingsboom van het brute force algoritme staan redundante knopen. Hiermee wordt bedoelt: vergelijkingen die altijd hetzelfde resultaat zullen opleveren. Bijvoorbeeld $E \times E+E+E$ en $E+E \times +E$. Echter moet er een verschil gemaakt worden tussen absoluut en relatief redundante knopen.%TODO Definitie in probleemstelling, hier of als sterretje beneden? Absoluut = schrappen ook op diepere niveaus, relatief = enkel op huidige niveaus bv ook E+E-E en E-E+E vanwege links toevoegen}%
In het geval van relatief redundante knopen mag de knoop nog niet verwijderd worden wanneer het volgende niveau nog moet berekend worden. In het geval van absoluut redundante knopen mag dit wel. Wanneer de redundante knopen verwijderd worden verkleint de zoekruimte, dit doet de uitwerkingstijd dalen. De oplossingsgraad blijft echter gelijk.
\subsubsection*{Het op voorhand berkenen van de bewerkingsboom}
Het zoeken van redundante knopen vraagt tijd. De bewerkingsboom veranderd niet zolang de CFG niet wijzigt. De mogelijkheid bestaat dus om de boom éénmalig op voorhand op te stellen. Een nadeel aan deze optie is dat het prunen van de uitgewerkte boom (zie verder) tijdrovender wordt. Uit experimenten blijkt dat de tijdswinst van het op voorhand berekenen niet opweegt tegenover het tijdsverlies. Dit komt voornamelijk doordat de tijdswinst op lage niveaus in de berekingsboom ($< 8$) niet groot is en er verwacht wordt dat de gebruiker geen vergelijkingen zoekt van lengte groter dan 6. Het zal later ook blijken dat deze te veel tijd vragen om te berkenen. 
\subsubsection*{Prunen in de uitwerking van de bewerkingsboom}
Ook in de uitwerking van de bewerkingsboom komen er redundante vergelijkingen naar boven. Zo zal de vergelijking $E+E-E$ in sommige gevallen redundant worden. Bijvoorbeeld wanneer ze wordt vervangen door $a+a-a$. Echter is ze niet altijd redundant, ze kan namelijk ook vervangen worden door $a+a-b$. Wanneer er vanuit gegaan wordt dat er constanten in de uitwerking voorkomen onstaan er nog meer redundaties. Zo zal de vergelijking $E+E$ ook terug te vinden zijn als $2*E$. De zoekruimte dus opnieuw verkleint zonder de oplossingsgraad te verlagen.
\subsubsection*{Prunen in realiteit}
Het is mogelijk om alle redundaties te verwijderen zonder de oplossingsgraad %TODO ander woord
te verlagen, dit proces vraagt echter veel rekenwerk en bijgevolg veel tijd. Daarom is er gekozen om sommige redundaties niet te verwijderen, sommige te bewaren en andere op eenvoudigere wijze uit te werken. Dit laatste puntje zorgt er echter voor dat er ook niet redundante knopen zullen verloren gaan. Bijvoorbeeld $a+1+1$ kan vervangen worden door $a+2$. Echter vraagt het zoeken van deze redundantie veel tijd. Algemener kan er gezegd worden dat er slechts één losstaande constante mag voorkomen. Het nadeel hiervan is dat het aantal constanten beperkt is, en bijvoorbeeld de vergelijking $a+100$ verloren zal gaan. Maar omdat het doel is te voldoen aan de gebruiker zijn verwachtingen en dus niet alle mogelijkheden te berekenen, is in dit geval de tijdswinst belangrijker dan de lichte daling in oplossingsgraad. %TODO ander woord
\subsubsection*{Opsomming prunemogelijkheden}
%TODO duidelijk maken dat + uitgevoerd geen verlies in oplossingen, * uitgevoerd met verlies in oplossingen, - niet uitgevoerd wegens tijdswinst niet genoeg tov tijdsverlies%
%TODO tabs voor uitlijning
+ Absoluut redundante knopen in bewerkingsboom\\
	bv. $E \ast E+E+E$ en $E+E \times E+E$\\
+ Onsplitsbare term %TODO ander woord%
links $\geq$ dan zijn rechterbuur\\
	bv. $a=2$ en $b=3$ dan mag $b+a$ voorkomen maar $a+b$ niet\\
+ Bewerkingen die bewerkingen ongedaan maken\\
	bv. $a+a-a$ of $a+a/a$ deze laatste kan vervangen worden door $a+1$\\
+ Neutrale constanten\\
	bv. $a \times 1$\\
- Relatief redundante knopen in bewerkingsboom\\
	bv. $E+E \times E$ en $E \times E+E$\\
$\ast$ Toelaten van slechts één losstaand gewicht\\
	bv. $a+1+1$ komt voor als $a+2$\\
$\ast$ Niet toelaten van zelfde onsplistbare term %TODO ander woord%\\
	bv. $a+a$ komt voor als $2 \times a$
\subsubsection*{Algoritme}
%TODO pseudocode


\end{document}