%%%% ijcai11.tex

\typeout{Eerste versie van de WV paper}

% These are the instructions for authors for IJCAI-11.
% They are the same as the ones for IJCAI-07 with superficical wording
%   changes only.

\documentclass{article}
% The file ijcai11.sty is the style file for IJCAI-11 (same as ijcai07.sty).
\usepackage{ijcai11}

% Use the postscript times font!
\usepackage{times}

% the following package is optional:
\usepackage{latexsym}
\usepackage{subfiles} 
\usepackage{amsmath}
\usepackage{fancybox}
\usepackage{varwidth}
\usepackage{cite}
\usepackage{framed}
\usepackage{graphics}
\usepackage{graphicx}
\usepackage{tabularx}
\usepackage{tikz}
\usepackage{tikz-qtree}
\usepackage{booktabs}
\graphicspath { {./subfiles/images/} }

% Allow to draw trees
\usepackage{caption}
\usetikzlibrary{trees}

% pseudocode\\
\usepackage{algorithm}
\usepackage{algpseudocode}
\usepackage{pifont}


% Use figuur ipv figure
\renewcommand{\figurename}{Figuur}

% Following comment is from ijcai97-submit.tex:
% The preparation of these files was supported by Schlumberger Palo Alto
% Research, AT\&T Bell Laboratories, and Morgan Kaufmann Publishers.
% Shirley Jowell, of Morgan Kaufmann Publishers, and Peter F.
% Patel-Schneider, of AT\&T Bell Laboratories collaborated on their
% preparation.

% These instructions can be modified and used in other conferences as long
% as credit to the authors and supporting agencies is retained, this notice
% is not changed, and further modification or reuse is not restricted.
% Neither Shirley Jowell nor Peter F. Patel-Schneider can be listed as
% contacts for providing assistance without their prior permission.

% To use for other conferences, change references to files and the
% conference appropriate and use other authors, contacts, publishers, and
% organizations.
% Also change the deadline and address for returning papers and the length and
% page charge instructions.
% Put where the files are available in the appropriate places.

\title{Flash fill en Equation Discovery}
\date{today}
\author{ 
 \begin{tabular}[t]{c@{\extracolsep{8em}}c} 
Craps Jeroen  & De Groote Tom \\
\textnormal{Bachelor Informatica} & \textnormal{Bachelor Informatica} \\
\textnormal{KU Leuven, Belgi\"e} &\textnormal{KU Leuven, Belgi\"e}  \\
\textnormal{jeroen.craps@student.kuleuven.be} & \textnormal{tom.degroote@student.kuleuven.be}
\end{tabular}
}

\begin{document}

\maketitle

\subfile{subfiles/Abstract.tex}
\subfile{subfiles/Introductie.tex}
\subfile{subfiles/Probleemstelling.tex}
\subfile{subfiles/Aanpak.tex}
\subfile{subfiles/Dataset.tex}
\subfile{subfiles/Experimenten.tex}
\subfile{subfiles/AndereAanpak.tex}
\subfile{subfiles/ToekomstigWerk.tex}
\subfile{subfiles/Conclusie.tex}
\subfile{subfiles/Code.tex}


\section*{Dankwoord}
Bij deze willen we graag onze begeleiders Prof. Luc De Raedt en Dr. Angelika Kimmig bedanken voor hun begeleiding en hulp gedurende dit project.


%% The file named.bst is a bibliography style file for BibTeX 0.99c
\bibliographystyle{named}
\bibliography{bibtex.bib}
\nocite{exampleLearning}
\nocite{spreadsheet}

%Make all appendixes after this title
\section*{Algoritme} \label{sec:algorimteBesprekeing}

\begin{algorithm*}[!htb]
\caption{Vind alle mogelijke vergelijkingen}
\label{VindVergelijkingen}
\begin{algorithmic}[1]
\Procedure{Search Equations}{}
\State $\textit{runOutOfTime} \gets \textit{false}$
\State $\textit{currentLevel} \gets \textit{CFG}$.getTerminalRulesAsEquations()
\State $\textit{nextLevel} \gets$new $\textit{List}$
\State $\textit{solutions} \gets$new $\textit{List}$
\While{$\neg\textit{runOutOfTime}$}
	\ForAll{$\textit{currentEquation} \gets \textit{currentLevel}$}
		\If{getRunOutOfTime()}
			\State $\textit{runOutOfTime} \gets \textit{true}$
			\State break
		\EndIf
		\ForAll{$\textit{nonTerminalRule} \gets \textit{CFG}$.getNonTerminalRules()}
			\ForAll{$\textit{terminalRule} \gets \textit{CFG}$.getTerminalRules()}
				\State $\textit{newEquation} \gets$ expand($\textit{currentEquation, nonTerminalRule, terminalRule}$)
				\If{isAbsoluteRedundant($\textit{newEquation}$)} 
					\State break
				\ElsIf{$\neg$shouldBePruned($\textit{newEquation}$)}
					\State $\textit{nextLevel}$.add($\textit{newEquation}$)
					\If{$\textit{newEquation}$.isGoalReached()} 
						\State $\textit{solutions}$.add($\textit{newEquation}$)
					\EndIf
				\EndIf

			\EndFor
		\EndFor
	\EndFor
	\State $\textit{currentLevel} \gets \textit{nextLevel}$
\EndWhile
\EndProcedure
\end{algorithmic}
\end{algorithm}




\end{document}